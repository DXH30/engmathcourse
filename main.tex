\documentclass{book}
\usepackage{subfiles}
\begin{document}
\chapter{Persamaan Diferensial Orde 1}
\par Dalam bab ini, kita akan mempelajari persamaan diferensial biasa yang diturunkan daripada masalah fisika atau lain-lain. Dalam menyelesaikannya menggunakan metod dasar, dan menafsirkan penyelesaiannya serta grafiknya  daam masalah yang diberikan itu. tentu saja, kita akan membicarakan tentang ketunggalan penyelesaian bagi  suatu persamaan diferensial peringkat pertama dalam bab ini juga.
\section{Persamaan Diferensial Biasa}
\par Persamaan Diferensial Biasa (PDB) merupakan persamaan yang hanya melibatkan turunan biasa bagi satu atau  lebih variabel tak bebas terhadap satu variabel bebas tunggal.
\begin{equation}
4 dy/dx - 5y = 1
\label{eq:pdb1}
\end{equation}
\par tedapat satu variabel tak bebas y dan satu variabel tak bebas x.
\begin{equation}
(x + y) dx - 9 y dy = 0
\end{equation}
\par terdapat satu variabel tak bebas y dan satu variabel tidak tak bebas x.
\begin{equation}
d^2y/dx^2 - 3 dy/dx + 5y = 0
\end{equation}
\par terdapat satu variabel tak bebas y dan satu variabel tidak tak bebas x.
\section{Persamaan Diferensil Parsial}
\par PDP adalah persamaan yang melibatkan turunan parsial bagi satu atau lebih variabel tak bebas terhadap dua atau lebih varabel bebas. Contoh-contoh persamaan dierensial parsial adalah seperti yang berikut :
\begin{equation}
{\del u}/{\del y} = - {\del v}/{\del x}
\end{equation}
\par Biasanya untuk diferensial parsial kita menggunakan tanda seperti :
\begin{equation}
{\del u}/{\del y}
\end{equation}
\par dan bukan lagi
\begin{equation}
du/dy
\end{equation}
sebab terdapat lebih dari satu variabel  beebas, dan dalam kassu diatas, variabel bebas adalah x dan y.
\begin{equation}
x^2 {\del u}/{\del x} + y {\del u}/{\del y} = u
\end{equation}
\chapter{Persamaan Diferensial Orde 2}
\chapter{Power Series}
\end{document}
